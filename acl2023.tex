% This must be in the first 5 lines to tell arXiv to use pdfLaTeX, which is strongly recommended.
\pdfoutput=1
% In particular, the hyperref package requires pdfLaTeX in order to break URLs across lines.

\documentclass[11pt]{article}

% Remove the "review" option to generate the final version.
\usepackage[review]{ACL2023}

% Standard package includes
\usepackage{times}
\usepackage{latexsym}

% For proper rendering and hyphenation of words containing Latin characters (including in bib files)
\usepackage[T1]{fontenc}
% For Vietnamese characters
% \usepackage[T5]{fontenc}
% See https://www.latex-project.org/help/documentation/encguide.pdf for other character sets

% This assumes your files are encoded as UTF8
\usepackage[utf8]{inputenc}

% This is not strictly necessary, and may be commented out.
% However, it will improve the layout of the manuscript,
% and will typically save some space.
\usepackage{microtype}

% This is also not strictly necessary, and may be commented out.
% However, it will improve the aesthetics of text in
% the typewriter font.
\usepackage{inconsolata}


% If the title and author information does not fit in the area allocated, uncomment the following
%
%\setlength\titlebox{<dim>}
%
% and set <dim> to something 5cm or larger.

\title{Instructions for ACL 2023 Proceedings}

% Author information can be set in various styles:
% For several authors from the same institution:
% \author{Author 1 \and ... \and Author n \\
%         Address line \\ ... \\ Address line}
% if the names do not fit well on one line use
%         Author 1 \\ {\bf Author 2} \\ ... \\ {\bf Author n} \\
% For authors from different institutions:
% \author{Author 1 \\ Address line \\  ... \\ Address line
%         \And  ... \And
%         Author n \\ Address line \\ ... \\ Address line}
% To start a seperate ``row'' of authors use \AND, as in
% \author{Author 1 \\ Address line \\  ... \\ Address line
%         \AND
%         Author 2 \\ Address line \\ ... \\ Address line \And
%         Author 3 \\ Address line \\ ... \\ Address line}

\author{First Author \\
  Affiliation / Address line 1 \\
  Affiliation / Address line 2 \\
  Affiliation / Address line 3 \\
  \texttt{email@domain} \\\And
  Second Author \\
  Affiliation / Address line 1 \\
  Affiliation / Address line 2 \\
  Affiliation / Address line 3 \\
  \texttt{email@domain} \\}

\begin{document}
\maketitle
\begin{abstract}
TBD
\end{abstract}

\section{Introduction}
In many customer-facing dialogue applications,
customer service interactions must follow a set of guidelines for safety,
which have a natural sequential order.
If a customer is locked out of their account and requests a password reset,
the agent must first verify that the customer is indeed the owner of the account.
This if-then structure is common to flows in guidelines.

Both human and robot agent must follow safety guidelines.
As a result, safety guidelines are often written in natural language.

Our goal is to train dialogue agents that not only follow a set of guidelines,
but justify their actions by pointing to the guidelines.
This allows others to verify their actions, and whether the guidelines have been followed.

We propose a generative model of dialogue,
that justifies decisions by aligning to a guidelines,
utilizes the sequential structure of guidelines,
and does not require supervision.

Experiments show that our model is accurate,
intepretable,
and works at a range of supervision levels.

Datasets include a variety of guidelines.
In ABCD, the guidelines are given to us \citet{abcd}.
In SGD, we write the guidelines ourselves,
using the generative model to aid development.
In doc2dial, we show that our method works for alignment to general document-guided
dialogue as well.

\section{Related work}
The adaptation of large languge models to task-oriented dialogue
has allowed for impressive results in zero-shot generalization,
where models are tested in scenarios that they have not previously seen
\cite{}.
The key idea behind this success is the use of a natural language interface:
specify scenario-specific details using natural language,
and take advantage of the generalization abilities of large language models.


\section{Problem setup}
We are interested in a generative model of dialogue that justifies its actions by
aligning to a document.
The model first chooses its alignments $z \sim p(z)$,

$p(x,z) = p(x\mid z)p(z)$


\bibliography{anthology,custom}
\bibliographystyle{acl_natbib}

\appendix

\section{Example Appendix}
\label{sec:appendix}

This is a section in the appendix.

\end{document}
